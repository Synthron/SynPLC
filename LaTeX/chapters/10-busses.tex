\chapter{Busses}

Every module has at least one common bus connector connecting via ribbon cable to the other modules. Alongside the 5V and GND voltage supply there also is the communication bus lines. On the modules there are two different busses possible: RS485 and CAN. 

In order to maintain a stable communication, consider the use of twisted-pair ribbon cables.

Both are working on a custom protocol, which is easy to implement on other controllers too. 

\section{RS485}

\subsection{Baudrate and Transmission speed}
The RS485 bus is transceiving at 250kBaud. At a maximum of 10 Bytes per frame, this results in a minimum rate of around 3000 frames per second.

\subsection{Frame Structure}
The frame structure of the protocol is as follows:
\vspace{0.5cm}

\begin{tabular}{cl}
    \textbf{Byte Number}     & \textbf{Data}                      \\
    1               & Startbyte (0x5A)          \\
    2               & Slave Adress              \\
    3               & Instruction and Register  \\
    4               & Data (Not used when Instruction == Scan or Read) \\
    5               & Data (only used at 12- or 16bit data blocks) \\
    6               & Checksum \\
    \\
    7               & Answer from Slave (ACK or NACK with Error code) \\
    8               & Answer from Slave (data, LSB at 12- or 16bit data blocks) \\
    9               & Answer from Slave (data, MSB at 12- or 16bit data blocks) \\
    10              & Answer from Slave (Checksum)
\end{tabular}
\vspace{0.5cm}

As you can see, most of the frame consists of the CPU talking to the slaves. Startbyte and Checksums are in place to make sure that an error free communication is possible. 

The Checksum itself is easily calculated. This is done by XOR-ing all sent bytes with the constant value 0xC5. In summary, the equation is as follows:

\begin{equation}
    Checksum = 0xC5 \oplus Byte\_1 \oplus Byte\_2 \oplus Byte\_3 (\oplus Byte\_4 \oplus Byte\_5)
\end{equation}

Equally, the slaves are calculating their checksum with the same method, XOR-ing all their bytes with 0xC5. 

If all the bytes are XOR-ed with each other and with the checksum itself and 0xC5 is still left, then the communication was error free and the data is uncompromized.

\subsection{Slave Addresses}
The slave addresses are an 8bit number which consist of a 3-bit module identifier and a 5-bit address settable via dip-switches on the respective module.

The module identifiers are definied as follows:

\begin{table}[H]
    \centering
        \begin{tabular}{|c|c|}
            \hline
            Identifier & Modul \\ \hline \hline
            000 & Digital Input \\ \hline
            001 & Digital Output \\ \hline
            010 & Analog Input \\ \hline
            011 & Analog Output \\ \hline
            100 & PWM Modul \\ \hline
            101 & Thyristor Modul \\ \hline
            110 & Relais Modul \\ \hline
            111 & \textit{reserved}    \\ \hline
        \end{tabular}
\end{table}

\subsection{Instructions}

In Byte 3 of the communication protocol are the instruction and the corresponding register specified. For more information on the registers, please refer to the corresponing module section in this document.

The instructions are at the 4 MSBs of the byte. They are as follows:

\begin{table}[H]
    \centering
        \begin{tabular}{|c|c|}
            \hline
            Instruction-Code & Instruction \\ \hline \hline
            0000 xxxx   & Bus Scanning (is slave available?) - no data byte \\ \hline
            0001 (reg)  & Read 8bit-Register\\ \hline
            0010 (reg)  & Write 8bit-Register \\ \hline
            0101 (reg)  & Read 16bit-Register \\ \hline
            0110 (reg)  & Write 16bit-Register \\ \hline
        \end{tabular}
\end{table}

\subsection{ACK, NACK and Error Codes}
In order to know weather the communication was successful, there is a byte used specifically for ACK/NACK and Error codes. This byte is divided into 2 nibbles, where the first nibble is the ACK/NACK and the second nibble is the error code. If everything is fine, the ACK nibble will be 0xC0 (1100). If there is an error present, it will display the NACK status with 0x03 (0011).

The following nibble will show which exactly went wrong:

\begin{table}[H]
    \centering
        \begin{tabular}{|c|c|}
            \hline
            Error Code & Meaning \\ \hline \hline
            0000 & No Error \\ \hline
            0001 & Target Register invalid \\ \hline
            0010 & Instruction invalid \\ \hline
            1100 & Processing Error - read Error Register \\ \hline
            1101 & Checksum Error - Send again \\ \hline
            1111 & System Error - read Error Register \\ \hline
        \end{tabular}
\end{table}