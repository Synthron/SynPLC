\chapter{Protokoll RS485}

\begin{tabular}{lll}
Frame & & \\
&	byte 1:& 	Startbyte (0x5A) \\
&	byte 2:&	Slave adress \\
&	byte 3:& 	Instruction and register \\
&	byte 4:& 	data (not send when instruction == Scan or read) \\
&	byte 5:& 	data (only send when writing 16bit (MSB) register) \\
&	byte 6:& 	Checksum \\
&&	\\
&	byte 7:& 	Answer from slave (ACK or (NACK $\vert$ Errorcode)) \\
&	byte 8:& 	Answer from slave (data, 8bit read or LSB when 16bit read) \\
&	byte 9:& 	Answer from slave (data, MSB when 16bit read) \\
&	byte 10:& 	Answer from slave (Checksum) \\
\end{tabular}


Hier wird das Protokoll der RS485-Schnittstelle beschrieben.

\section{Genereller Aufbau}

\subsection{Baudrate}

Der RS485-Bus wird mit einer Baudrate von 250.000 Baud betrieben.

\section{Protokoll}
\subsection{Ablauf}
Je gesamtem Frame (eine erfolgreiche Datenübertragung) werden 3-5 Byte übertragen. Dieses Frame setzt sich wie folgt zusammen:
\begin{itemize}
    \item von CPU:
    \begin{enumerate}
        \item Slave-Adresse (1 Byte)
        \item Befehl und Register (1 Byte)
        \item Daten (1 oder 2 Byte)
    \end{enumerate}
    \item von Slave
    \begin{enumerate}
        \item ACK und Fehlercode (1 Byte)
    \end{enumerate}
\end{itemize}

Je nach Slave können mehrere Frames für die vollständige Datenübertragung vonnöten sein. 

\subsection{Slave-Adresse}
Die Slave-Adresse besteht aus 2 Teilen:
\begin{itemize}
    \item Modul-Identifier (3 bit)
    \item Adresse ( 5 bit)
\end{itemize}

Der Modul-Identifier ist in Software festgelegt und bezeichnet die Art des Moduls:

\begin{table}[H]
    \centering
        \begin{tabular}{|c|c|}
            \hline
            Identifier & Modul \\ \hline \hline
            000 & Digital Input \\ \hline
            001 & Digital Output \\ \hline
            010 & Analog Input \\ \hline
            011 & Analog Output \\ \hline
            100 & PWM Modul \\ \hline
            101 & Thyristor Modul \\ \hline
            110 & Relais Modul \\ \hline
            111 & \textit{reserviert}    \\ \hline
        \end{tabular}
\end{table}

Die 5bit-Adresse wird über die Codierschalter am Slave vorgenommen.

\subsection{Befehl und Register}

Die Befehle und Register werden in einem Byte zusammengeführt. Beide Teile sind je 4 bit lang. Die Befehle setzen sich dabei wie folgt zusammen:

\begin{table}[H]
    \centering
        \begin{tabular}{|c|c|}
            \hline
            Befehl-Code & Befehl \\ \hline \hline
            0000 xxxx   & Bus Scanning (ist der Slave anwesend?) - kein Datenbyte \\ \hline
            0001 (reg)  & Lese 8bit-Register\\ \hline
            0010 (reg)  & Schreibe 8bit-Register \\ \hline
            0101 (reg)  & Lese 16bit-Register \\ \hline
            0110 (reg)  & schreibe 16bit-Register \\ \hline
        \end{tabular}
\end{table}

Die Register sind dabei modulabhängig. Folgend sind die Register aufgelistet:

\subsubsection{Digital Output}
\begin{table}[H]
    \centering
        \begin{tabular}{|c|c|}
            \hline
            Registeradresse & Funktion \\ \hline \hline
            0001 & Output-Register \\ \hline
            0010 & Input-Register \\ \hline
            1111 & Fehlerregister \\ \hline
        \end{tabular}
\end{table}

\subsubsection{Digital Input}
\begin{table}[H]
    \centering
        \begin{tabular}{|c|c|}
            \hline
            Registeradresse & Funktion \\ \hline \hline
            0010 & Input-Register \\ \hline
            1111 & Fehlerregister \\ \hline
        \end{tabular}
\end{table}

\subsubsection{Analog Output}
\begin{table}[H]
    \centering
        \begin{tabular}{|c|c|}
            \hline
            Registeradresse & Funktion \\ \hline \hline
            0001 & Channel-Enable-Register \\ \hline
            0010 & Spannung Channel 1 (16 bit) \\ \hline
            0011 & Strom Channel 1 (16 bit)\\ \hline
            0100 & Spannung Channel 2 (16 bit)\\ \hline
            0101 & Strom Channel 2 (16 bit)\\ \hline
            0110 & Spannung Channel 3 (16 bit)\\ \hline
            0111 & Strom Channel 3 (16 bit)\\ \hline
            1001 & Spannung Channel 4 (16 bit)\\ \hline
            1010 & Strom Channel 4 (16 bit)\\ \hline
            1100 & Channel-Status-Register \\ \hline
            1101 & Channel-Fehler-Register \\ \hline
            1111 & Fehlerregister \\ \hline
        \end{tabular}
\end{table}

\subsection{ACK und Fehlercode}
Damit sichergestellt werden kann, dass die Daten alle richtig empfangen wurden, werden ACK (Acknowledge) und Fehlercodes gesendet. Diese werden in einem Byte zusammen geschickt und bestehen aus je 4bit.

Ein positives Acknowledge wird mit dem Hex-Code 5 (0101) codiert, ein negatives Acknowledge (NACK) wird mit A (1010) codiert. die Fehlercodes setzen sich dabei wie folgt zusammen:

\begin{table}[H]
    \centering
        \begin{tabular}{|c|c|}
            \hline
            Fehlercode & Bedeutung \\ \hline \hline
            0000 & Kein Fehler \\ \hline
            0001 & Ziel-Register nicht bekannt \\ \hline
            0010 & Befehl nicht bekannt \\ \hline
            1100 & Prozessfehler - Fehlerspeicher auslesen \\ \hline
            1101 & Prüfsummenfehler - Sende erneut \\ \hline
            1111 & Systemfehler - Fehlerspeicher auslesen \\ \hline
        \end{tabular}
\end{table}