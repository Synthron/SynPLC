\chapter{Bus-Systeme}
Die CPU unterstützt nativ zwei Bus-Physiken:
\begin{itemize}
    \item RS485
    \item CAN
\end{itemize}

Für die IO-Module V1 wird nur RS485 unterstützt. Dazu wird auf der CPU die Steckbrücke für den Backplane entsprechend gesetzt.

%\section{RS485}
%Der RS485-Bus ist für die IO-Module der V1 vorgesehen. Dieser ist einfach auch auf kleineren und günstigeren Controllern zu nutzen und bietet genug Sicherheit über längere Strecken durch seinen differentiellen Aufbau.

%Ursprünglich war angedacht, hier das ModBus RTU Protokoll zu nutzen, allerdings wurde dagegen entschieden und ein eigenes Protokoll entwickelt, was direkter ist und einige Funktionen mehr unterstützt. Auch sind die Funktionen auf die IO-Module zurechtgeschnitten.

%\subsection{Adressen}
%Es wird ein 8-bit Adressverfahren genutzt. Dies passt in die seriellen Buffer der Controller. Außerdem ist nicht vorgesehen, mehr als 32 Geräte eines Typs gleichzeitig zu benutzen. Damit die CPU von sich aus weiß, welches Modul auf welcher Adresse ist, sind die Adressen kodiert. 

%\begin{table}[h]
%    \centering
%\begin{tabular}{|c|c|}
%    \hline
%    Modul       &   Adressbereich   \\ \hline \hline
%    CPU         &   111X XXXX       \\ \hline
%    Inferface   &   001X XXXX       \\ \hline
%    Digital Out &   010X XXXX       \\ \hline
%%    Digital In  &   011X XXXX       \\ \hline
%    Analog Out  &   100X XXXX       \\ \hline
%    Analog In   &   101X XXXX       \\ \hline
%    PWM         &   110X XXXX       \\ \hline
%\end{tabular}
%\caption{Adressen RS485}
%\label{tab:Adressen RS485}
%\end{table}